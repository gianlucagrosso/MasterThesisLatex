\chapter{Results}
\label{chap:results}
\note{Plot ideas}{
\begin{itemize}
    \item Plot for $N \geq 2$ functions where I know analytic $f$: plot $f(x_1,..., x_N)$ vs $x_1$ where $x_2,..., x_N$ are picked randomly
    \item plot bond dimension at varying $R$ on the same plot (see Fig. 4 \cite{Fernandez2024})
\end{itemize}
}
\section{Approximation of Green's functions}

\begin{figure}[ht!]
    \caption{Error Heatmap of Green's function.}
\end{figure}

\begin{figure}[ht!]
    \caption{$G_{exact}(x,y)$ \& $G_{approx}(x,y)$ vs $x$ for $y$-points picked at random.}
\end{figure}

\begin{figure}[ht!]
    \caption{Memory and time vs $\delta$ (localization parameter) patched and non-patched.}
\end{figure}

\begin{figure}[ht!]
    \caption{Global error $\epsilon$ vs $R$ to check convergence.}
\end{figure}

\section{BSE with Patching}

\subsection{\texttt{automul} algorithm}

\begin{figure}[ht!]
    \caption{Algorithm flowchart of matrix + element-wise multiplication\\(\texttt{automul}).}
\end{figure}

\begin{figure}[ht!]
    \caption{Heatmap of TCI and patched TCI approximation of the BSE vertices with patch grid (2D slices or multiple 2D sliced for 3D figure )}
\end{figure}

\begin{figure}[ht!]
    \caption{Memory vs patch bond dimension comparison patched/non-patched (show sub-optimal approximation)}
\end{figure}

\begin{figure}[ht!]
    \caption{N patches vs patch bond dimension with optimal number of patches reference for matrix muliplication for each BSE vertex (highlight chosen data for contraction)}
\end{figure}

\begin{figure}[ht!]
    \caption{Memory and time scaling patched/non-patched BSE contraction vs $R$ and $\epsilon$.}
\end{figure}

\section{Bubble Calculation}

\note{Papers}{Convolution in frequency domain \cite{Rakhuba2015}}

\begin{figure}[ht!]
    \caption{Heatmap and scheme of the global computation (TCI+QFT+ element-wise contraction + patch sum + $\textrm{QFT}^{-1}$ )}
\end{figure}

\begin{figure}[ht!]
    \caption{Memory and time scaling vs function parameter ($\beta$, $R$, $\epsilon$) comparison patch/non-patch }
\end{figure}


\section{Gaussian Orbital Overlap}

\note{Orbitals overlap}{Yuriel's paper on atomic bases: \cite{Jolly2024}, double-zeta calculation(?)}

The electron repulsion integral of two GTOs is defined as

\begin{equation}
    \bra{\textbf{A}, \textbf{C}}\frac{1}{r_{12}}\ket{\textbf{B}, \textbf{D}} = \int_{-\infty}^\infty \text{d}^3r_1 \int_{-\infty}^\infty \text{d}^3r_2 \frac{\phi_\textbf{A}(\bm{r}_1) \phi_\textbf{B}(\bm{r}_1) \phi_\textbf{C}(\bm{r}_2) \phi_\textbf{D}(\bm{r}_2)}{|\bm{r}_1 - \bm{r}_2|}
\end{equation}

where the generic Cartedisan GTO can be written as

\begin{equation}
    \ket{\textbf{R}} = \phi_\textbf{R}(\bm{r}) = N (x - R_x)^l (y - R_y)^m (z - R_z)^n e^{-\alpha (\bm{r} - \textbf{R})^2}.
\end{equation}
The electron repulsion integral can be rewritten as (cf. Ref. \cite{Petersson2010}): 

\begin{align}
    \bra{\textbf{A}, \textbf{C}}\frac{1}{r_{12}}\ket{\textbf{B}, \textbf{D}} =& \frac{N_\textbf{A}N_\textbf{B}N_\textbf{C}N_\textbf{D} \pi^{5/2}}{\gamma_p\gamma_q\sqrt{\gamma_p + \gamma_q}} e^{\eta_p(\textbf{A} - \textbf{B})^2} e^{\eta_q(\textbf{C} - \textbf{D})^2} \times \\
    \nonumber &\sum_{\bm{i},\bm{o},\bm{r},u}\mathcal{J}_x \sum_{\bm{j},\bm{p},\bm{s},v} \mathcal{J}_y \sum_{\bm{k},\bm{q},\bm{t},w} \mathcal{J}_z\ 2F_\nu (\eta (\textbf{P} - \textbf{Q})^2)
\end{align}

where

\begin{equation} 
    F_\nu(u) = \int_{0}^{1} \text{d}t\ t^{2\nu} e ^{-ut^2}
\end{equation}

is the so-called \textit{Boys function} \cite{Boys1950}.

    
\begin{equation}
    \begin{alignedat}{5}      
      \textbf{P} &= \frac{1}{\gamma_p}(\alpha_1 \textbf{A} + \alpha_2 \textbf{B}) &\qquad \gamma_p &= \alpha_1 + \alpha_2 \\[6pt]
      \textbf{Q} &= \frac{1}{\gamma_q}(\alpha_3 \textbf{C} + \alpha_4 \textbf{D}) &\qquad \gamma_q &= \alpha_3 + \alpha_4 \\[6pt]
      \eta &= \frac{\gamma_p\gamma_q}{\gamma_p + \gamma_q}
    \end{alignedat}
\end{equation}
    

\begin{figure}[ht!]
    \caption{TCI and patched TCI approximation of the Boys function.}
\end{figure}

\begin{figure}[ht!]
    \caption{3D graph of Gaussian orbitals for \ce{H2} and \ce{LiH}}
\end{figure}

\begin{figure}[ht!]
    \caption{Memory and time scaling of orbital overlap compared with standard approach. }
\end{figure}



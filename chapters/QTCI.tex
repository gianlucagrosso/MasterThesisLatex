\chapter{Quantics Tensor Cross Interpolation}
\label{chap:QTCI}


\section{The algorithm}

\begin{definition}[Compressible tensor]
	A tensor $\mathcal{T}$ is {\normalfont \textbf{compressible}} or {\normalfont \textbf{low-rank}} if it can be approximated by a Matrix Product State (MPS) with small rank $\chi$.
	\label{def:compresstensor}
\end{definition}

\begin{definition}[Rank-revealing algorithm]
	An algorithm 
	
	\[
		\renewcommand{\arraystretch}{1.1}% a touch of extra vertical room
		\begin{array}{r c >{{}}c<{{}} c} 
		\mathcal{A}: &
		\mathds{K}^{I_1 \times \dots \times I_\mathcal{L}} &
		\longrightarrow &
		\mathds{K}^{I_1 \times \dots \times I_\mathcal{L}} \\[2pt]
		& \mathcal{T} &
		\longmapsto &
		\widetilde{\mathcal{T}}
		\end{array}
	\]

	is said to be {\normalfont \textbf{rank-revealing}}, if it ouputs a low-rank approximation $\widetilde{\mathcal{T}}$ for any compressible $\mathcal{L}$-dimensional tensor $\mathcal{T}$\footnotemark given as input.
	\label{def:rkralg}
\end{definition}

\footnotetext{In \prettyref{def:rkralg} we define a tensor $\mathcal{T}$ as an element of the vector space $\mathds{K}^{I_1 \times \dots \times I_\mathcal{L}}$, this is only true if we consider $\mathcal{T}$ an $\mathcal{L}$-dimensional numerical array, as it is for numerics. More generally $\mathcal{T} \in T^p_q(V) = \{t\ |\ t:V^{\otimes q} \otimes  (V^*)^{\otimes p} \longrightarrow \mathds{K} \} $ (where $\mathds{K} = \mathds{R} \vee \mathds{C}$ and $V = \mathcal{H}$ for most applications).}


\subsection{Matrix Cross Interpolation and prrLU}
The TCI algorithm bases its implementation on the following statement: \textit{a low-rank matrix is strongly compressible. }

\begin{example}[$5 \times 5$ Correlation matrix]
For classical Harmonic Oscillator chain mode-like vectors: 

\[
\begin{alignedat}{2}      
	\boldsymbol{v} = \frac{\boldsymbol{1}_5}{\sqrt{5}} \qquad \boldsymbol{w} = \frac{(-2,-1,\dots,2)}{2\sqrt{5}}
\end{alignedat}
\]
the correspondent position-position correlation matrix can be Cross Interpolated as 
\begin{equation}
	\label{eq:CIcorrmat}
	\begin{aligned}
	  & C  \;=\; \sigma^2 \boldsymbol{v}^T \boldsymbol{v} + \tilde{\sigma}^2 \boldsymbol{w}^T\boldsymbol{w} = \\[.4em]
	  & {}=  \underbrace{%
	  \mbox{\footnotesize$
	  \frac{1}{115}\left[\begin{array}{ccccc}
		23\sigma^2 + 72\tilde{\sigma}^2  & 23\sigma^2 - 18\tilde{\sigma}^2 & 23\sigma^2 + 12 \tilde{\sigma}^2  & 23\sigma^2 - 18\tilde{\sigma}^2 & 23\sigma^2 - 48\tilde{\sigma}^2 \\[0.2em]
		\cdots  & 23\sigma^2 + 9\tilde{\sigma}^2/2 & 23\sigma^2 - 3\tilde{\sigma}^2 & 23\sigma^2 + 9\tilde{\sigma}^2/2 & 23\sigma^2 + 12\tilde{\sigma}^2 \\[0.2em]
		\cdots & \cdots & 23\sigma^2 + 2\tilde{\sigma}^2 & 23\sigma^2 - 3\tilde{\sigma}^2 & 23\sigma^2 - 8\tilde{\sigma}^2 \\[0.2em]
		\cdots  & \cdots & \cdots &  23\sigma^2 + 9\tilde{\sigma}^2/2 & 23\sigma^2 + 12\tilde{\sigma}^2 \\[0.2em]
		\cdots  & \cdots & \cdots & \cdots & 23\sigma^2 + 32\tilde{\sigma}^2
	  \end{array}\right]$}
	  }_{\displaystyle 5 \times 5}
	  \\[1.2em]
	  & {}\approx
	\underbrace{%
	  C[\mathds{I} , \mathcal{J}]}_{\displaystyle 5 \times 2}
	\,
	\underbrace{%
	  \begin{bmatrix}
		\frac{23\sigma^2 + 72\tilde{\sigma}^2}{115} & \frac{23\sigma^2  -48 \tilde{\sigma}^2}{115} \\[0.2em]
		\frac{23\sigma^2 - 48\tilde{\sigma}^2}{115} & \frac{23\sigma^2 + 32\tilde{\sigma}^2}{115} 
	  \end{bmatrix}^{-1}}_{\displaystyle 2 \times 2}
	  \,
	  \underbrace{%
	  C[\mathcal{I} , \mathds{J}]}_{\displaystyle 5 \times 2}
	\end{aligned}
\end{equation}
with $\sigma$ and $\tilde{\sigma}$ the variances of the two modes and $\mathcal{I} = \{ 1,5\}$ and $\mathcal{J} = \{ 1,5\}$. 
\end{example}

	

\note{Matrix Cross Interpolation}{Main paper: \cite{Goreinov1997, Schneider2010}, Error is at most $O(\chi^2)$ the optimal error of SVD, CI vs prrLU, exact for the selected rows and columns (\textit{interpolation}) $A(\mathcal{I},\mathcal{J})*A(\mathcal{I},\mathcal{J})^{-1}A(\mathcal{I},\mathds{J}) = A(\mathcal{I},\mathds{J})  $ and for $\rank{A} = \chi$, heuristic \textit{maxvol principle} = maximise determinant of the pivot matrix, using continuous MCI already reduces the complexity of integration and derivation, CI is numerically unstable for large $\chi$}\vspace{\baselineskip}\\

\noindent\note{Nice Sentences}{The RHS contains only small subparts of the original matrix. }\\

\noindent\note{prrLU}{Main book: \cite{Golub96}}

\noindent\todo{Insert a numerical example of CI and prrLU $\to$ use the DFT matrix}


\subsection{Tensor Cross Interpolation}

The number of function calls necessary to construct $\widetilde{\mathcal{T}}$ from $\mathcal{T}$ is $O(\mathcal{L}d\chi^2) \ll d^\mathcal{L}$, fully specifying only $O(\mathcal{L}\chi^2)$ number of \textit{pivots}.

The TCI algorithm computational time scales as $O(\mathcal{L}d\chi^2)$. 

\begin{definition}
	A function $f$ is {\normalfont \textbf{almost separable}} if its tensor approximation $F$ is \underline{low-rank}.
	\label{def:separfunc}
\end{definition}


\note{Tensor Cross Interpolation}{Main paper: \cite{Fernandez2024}, Main TCI ref: \cite{Oseledets2010}, polynomial dimension scaling, matrix cross interpolation \cite{Goreinov1997}, function learning, N-dimensional integrals, \textit{$\epsilon$-factorizability} of continuous functions $\to$ reduction from N to one-dimensional integrals independent of function sign ($\int \text{d}\mathbf{x} f(x_1, ... , x_N) \approx \prod\limits_{\alpha = 1 }^N\int \text{d}x_1 T_\alpha(x_\alpha)P_\alpha^{-1}$) and $|| f - f_\text{TCI} || < \epsilon $, N-dimensional integral $O(\mathcal{L}d\chi^2)$ (not $d^\mathcal{L}$) , tensor representation $O(\mathcal{L}\chi^2)$ (??), \textit{active machine learning} \cite{Settles2012}: find the region with the largest approximation error, main challenge: bookkeeping and notation, \textit{maxvol} principle \cite{Dolgov2020}, exact if $\rank{TT} = \chi$ like MCI (see Naive Approach \cite{Fernandez2022}), \textit{multi-indices} and \texttt{Vector}\{\textit{multi-indices}\}, $\bigoplus$= concatenation of multi-indices, \textit{nesting condition} \cite{Oseledets2011,Dolgov2020}$\to$ $\mathcal{I}_\alpha \subset \mathcal{I}_{\alpha -1 } \bigoplus \mathds{K}_\alpha$ and $\mathcal{J}_\alpha \subset \mathds{K}_\alpha \bigoplus \mathcal{J}_{\alpha +1 }$(this guarantees exact interpolation for $f$(pivots)), function calls: $O(\mathcal{L}d\chi^2) \ll d^\mathcal{L}$, $\Pi_\alpha$ and the \textit{error function} $\epsilon_\Pi = |f - f_\text{TCI}|(i, x_\alpha, x_{\alpha +1}, j)$ $\to$ find the maximum of $\epsilon_\Pi$ before adding new pivots (add the pivots that yields the largest improvement of the local accuracy), pivot search: \textit{full}, \textit{rook}, \textit{block-rook}, \textit{alternate}(alternate:single pivot $O(d\chi)$, total search $O(d\chi^2)$ \cite{Fernandez2022}), \textit{environment-aware} $\epsilon_\Pi$ $\to$ error function related to the error of the integral that takes in consideration the volume contribution to the integral, \textit{environment} vectors $L_i$ and $R_j$, \textit{rank-revealing} (only slow convergence for high rank), \textit{polynomial costs}, integration scales $1/N_{eval}$ compared to $1/\sqrt{N_{eval}}$ Monte Carlo (cf. Ref. \cite{Fernandez2024})}\\

\noindent \todo{Insert Algorithm description (see below)}
\begin{algorithm}
\caption{An algorithm with caption}\label{alg:cap}
\begin{algorithmic}
\If{something}
    \State do something else
\EndIf
\end{algorithmic}
\end{algorithm}

\note{Applications}{Feynmann diagramm expansion \cite{Fernandez2022}: high-order real-time nonequilibrium Schwinger-Keldysh perturbation expansions (integral convergence $1/N^2$, $N$ function evaluations), multi-dimensional function minimization and quantized reinforcement learning \cite{Sozykin2022} (outperfoming standardized gradient-free methods in number of function evaluations and execution time), computation of Brilloin zone integrals to calculate topological invariants \cite{Ritter2024}, compact tensorization of atomic orbitals bases with high accuracy \cite{Jolly2024} (error on energy of \ce{H2} improved by 85\% w.r.t. double zeta calculation), speed-up of multi-assets Fourier transform-based option pricing \cite{Sakurai2025}}\\

\noindent\note{Nice sentences}{The approximation is systematically controlled by $\chi$. TCI is a generalization of the matrix cross interpolation to $N$-dimensional tensors and functions. Summation is implicit over the indices connecting two  tensors. TCI representation is defined by the selected $\mathcal{I}$ and $\mathcal{J}$, so an accurate interpolation amounts to optimizing this selection. TCI approximation of $f$ is built from one-dimensional slices  of $f$. Tensor network methods offer a new appraoch to high-dimensional integration. It significantly outperform Monte-Carlo and quasi-Monte-Carlo methods in differente applications. The limiting factor of TCI (rank of the $\epsilon$-factorization) is entirely orthogonal to that of sampling methods. The tensor cross interpolation (TCI) algorithm is a rank-revealing algorithm for decom-
posing low-rank, high-dimensional tensors into tensor trains/matrix product states (MPS).}\\

\noindent\note{General references}{\cite{Oseledets2010, Oseledets2011, Savostyanov2014, Savostyanov2011, Dolgov2020}}\\
\note{Open questions}{Generality of $\epsilon$-factorizability property}

\subsection{The Quantics Representation: QTCI}



\note{QTCI}{Discards weak entanglement between different scales; works well for scale-separated, non-irregular functions; $O(\mathcal{L} \log N )$ ($N \stackrel{?}{=} 2^R $) quantics approximation scaling 4 $\to$ logarithmic scaling in the number of grid points \cite{Khoromskij2011}; exponentially high resolution; low-rank scale separated structure revealed through TCI + quantics }\\

\noindent \note{Nice Sentences}{Numerical computations with multivariate functions typically involve a compromise between two contrary desiderata: accurate resolution of the functional dependence, versus parsimonious memory usage.}\\

\noindent \note{General references}{\cite{Oseledets2009} (quantics and interleaved representation) }\\

\noindent \note{Applications}{Approximation and manipulation of correlation functions for quantum many body systems \cite{Hiroshi2023}; compression of imaginary-time propagators in the Frobenious norm \cite{Takahashi2025}; diagrammatic non-equilibrium many-body Green’s function-based calculations \cite{Murray2024} $\to$ better scaling of resourses for higher precisions $R$ or longer times }\\

\noindent \note{Open questions 
}{When does a multivariate function admit low-rank representation? }





\section{Computational Fallbacks: Localised Functions}

\begin{figure}[ht!]
	\caption{2D: Heatmap + error Heatmap of TCI approximation of a very localised function with different zoomings  }
\end{figure}

\subsection{Localised Function and Memory Consumption}

\begin{figure}[ht!]
	\caption{1D: plot of different patch subdivision and relative bond dimension per patch for each division (like patching paper)  }
\end{figure}

\begin{figure}[ht!]
	\caption{2D: Bond dimension vs bond for each of the zoomings and for one of the more trivial parts of the function. }
\end{figure}


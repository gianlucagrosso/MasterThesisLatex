\chapter{Patched QTCI}
\label{chap:patching}

\note{Patching}{Main reference \cite{Hiroshi2023}}\\

\noindent \note{General references }{\cite{Murray2024} (first mention of the patching+QTT scheme)}\\ 

\section{The algorithm: the Patching scheme}

\begin{figure}[h!]
	\caption{Tensor representation of a projected MPS and of a container of projected MPSs (patched MPS)}
\end{figure}

\begin{figure}[h!]
	\caption{Small flowchart of the patching algorithm.}
\end{figure}
\section{Computational Costs and Scaling}

\begin{figure}[h!]
	\caption{Memory and time scaling vs patch bond dimension + fit (theoretical values)}
\end{figure}

\begin{figure}[h!]
	\caption{Number of patches vs patch bond dimension + fit }
\end{figure}

\begin{figure}[h!]
	\caption{Heatmap of patched approximation with patch grid for different patch bond dimension and tolerances.}
\end{figure}

\begin{figure}[h!]
	\caption{Plot of optimal patching (number of patches) region for function approximation (number of patches vs bond dimension per patch)}
\end{figure}



\section{Applications}

\subsection{Functions with Localised Features}

\begin{figure}[h!]
	\caption{Heatmap + error Heatmap of patched TCI approximation of a very localised function with different zoomings  }
\end{figure}

\begin{figure}[h!]
	\caption{Bond dimension vs bond for the different patches. }
\end{figure}




% \lipsum
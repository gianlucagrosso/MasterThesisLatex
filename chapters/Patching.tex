\chapter{Patched QTCI}
\label{chap:patching}

\note{Patching}{Main reference \cite{Hiroshi2023}; \textit{reset mode} of TCI eliminates bad pivots that occur when the algorithm first explores regions where $\mathcal{T}$ is small and only later configurations that are larger $\to$ this could be sufficient but not enough hence patching; patched TCI benefints from \textit{global pivot updates} coming from previous TCI decompositions; patching helps solving ergodicity problems; }\\

\noindent \note{General references }{\cite{Murray2024} (first mention of the patching+QTT scheme)}\\ 

\noindent \note{Plot ideas}{Plot the ratio of sampled points (function calls) vs total points of the calculation. }\\ 

\noindent \note{Nice sentences}{ Nevertheless,
since all TCI algorithms involve sampling, none of them is fully immune against missing some features of the tensor of interest, as already discussed above.}

\section{The algorithm: the Patching scheme}
\label{sec:patchAlg}

\begin{figure}[ht!]
	\caption{Tensor representation of a projected MPS and of a container of projected MPSs (patched MPS)}
\end{figure}

\subsection{Direct sum of Tensor Trains}

\begin{figure}[ht!]
	\caption{Small flowchart of the patching algorithm.}
\end{figure}
\section{Computational Costs and Scaling}

\begin{figure}[ht!]
	\caption{Memory and time scaling vs patch bond dimension + fit (theoretical values)}
\end{figure}

\begin{figure}[ht!]
	\caption{Number of patches vs patch bond dimension + fit }
\end{figure}

\begin{figure}[ht!]
	\caption{Heatmap of patched approximation with patch grid for different patch bond dimension and tolerances.}
\end{figure}

\begin{figure}[ht!]
	\caption{Plot of optimal patching (number of patches) region for function approximation (number of patches vs bond dimension per patch)}
\end{figure}

\subsection{``Overpatching''}
\begin{figure}[ht!]
	\caption{1D: plot of different patch subdivision and relative bond dimension per patch for each division (like patching paper)  }
\end{figure}



\section{Applications}

\subsection{Functions with Localised Features}

\begin{figure}[ht!]
	\caption{Heatmap + error Heatmap of patched TCI approximation of a very localised function with different zoomings  }
\end{figure}

\begin{figure}[ht!]
	\caption{Bond dimension vs bond for the different patches. }
\end{figure}




% \lipsum
\chapter{Introduction}
\label{chap:intro}
Tensor network techniques are a well-established \cite{Fannes1992,PhysRevLett.69.2863,SCHOLLWOCK201196,PhysRevLett.91.147902,verstraete2004renormalizationalgorithmsquantummanybody,vonDelftTNNotes,tensornetwork.org} and standard approach widely employed by the quantum physics community to mitigate the exponential increase of computational resources encountered when solving high-dimensional quantum and many-body problems.\vspace{\baselineskip}\\

\note{Tensor Networks}{\textit{MPS or TT toolset} for reducing the rank of tensor object}

\textsc{Small caps text}
{\HUGE{Huge text}}


Some different type of text:\hfill

Mixing \textbf{different series, \textsf{families}} and
\textsl{\texttt{shapes,}} \textsc{especially in one sentence,}
is usually \emph{highly inadvisable!}\textit{ it = emph}

Some modification 

\MakeTextUppercase{⟨text⟩}


% \begin{figure}
% 	\centering
% 	\includegraphics[width=.5\textwidth]{example-image-duck}
% 	\caption{A cool duck}
% 	\label{fig:figure1}
% \end{figure}



% \label{sec:outline}

% A cool list:
% \begin{enumerate}
% 	\item item 1
% 	\item item 2
% 	\item item 3
% \end{enumerate}

% \rem{Something}
%  \todo{I have to fix this}
%  \note{Note }{just a note}

% \subsubsection{Some section}
% Some text
% \begin{definition}[Some law]
%     Some details
% \end{definition}
% Thank you for the space
% \change{\note{Gianluca}{I have to do this}}

% \begin{mydef}[Title]
%     Some text
% \end{mydef}
    

% $\Tr (\Gamma)$








\chapter{Introduction}
\label{chap:intro}
``Dispelling'' the \textit{curse of dimensionality} that ``hexes'' numerics' computations, has been a challenge of prime interest in different fields of science for many years. 
Tensor network techniques -- and in particular \textit{matrix product states} (MPSs) methodologies -- are a well-established \cite{Fannes1992,White1992,Schollwock2011,Vidal2003,VerstraeteCirac2004,vonDelftTNNotes,tensornetwork.org} and standard approach widely employed by the quantum many-body (QMB) physics community to mitigate such exponential increase of computational resources. Numerical simulations of high-dimensional QMB wavefunctions represent, in fact, an exceptionally challenging computational task if not properly addressed.

Mathematicians have dedicated, as well, significant effort to target the same problem -- arising from multidimensional tensor approximation of continuous functions or complex numerical linear algebra computations \cite{Oseledets2009Intro, Kolda2009} -- ultimately converging towards a similar approach: \textit{tensor trains} (TTs)\footnote{From now on \textit{tensor train} and \textit{matrix product state} will be used interchangeably.} \cite{Oseledets2011}. 

The rising interest in tensor-network (TN) methods -- particularly those based on MPSs — has therefore led to the emergence of a standard ``MPS toolbox'' \cite{ttpylib, ITensors.jl, QSpace} and a well-defined catalogue of canonical TN applications \cite{Verstraete2008} (variational ansatz of QMB wavefunctions -- or DMRG -- among the most popular in physics). For a long time, however, techniques for function approximation and manipulation have been laying beyond the traditional TN portfolio.


Classical numerical techniques for representing and manipulating functions--whether for integration, convolution, differentiation, or related tasks--have advanced considerably over the years \cite{Isaacson1994}, yet they remain hindered by significant constraints. Grid‐based or naive SVD–style tensor discretizations of multivariate functions, for instance, confront the \emph{curse of dimensionality}: the memory and CPU time required to store and update a high‐resolution representation of a $\mN$-dimensional function grow exponentially with 
$\mN$. When it comes to integration, standard stochastic approaches such as Monte Carlo and Quantum Monte Carlo fare no better in the high‐dimensional regime; their error decreases only algebraically with sample size and they are plagued by additional obstacles—most notably the “sign problem’’ \cite{Loh1990}—that can render simulations impractical for large, strongly correlated systems \cite{Fernandez2022}. Together, these limitations leave many modern, high-dimensional applications beyond the reach of standard numerical methods.

The widespread and increasing interest in TN and MPSs methodologies among different fields has facilitated the development of a pivotal\footnote{A rather playful wording choice given the context.} extension to the ``MPS toolbox'' in order to integrate the missing function representation capabilities: the Tensor Cross Interpolation (TCI) algorithm.

TCI attempts to address many of the above-mentioned limitations of standard function approximation and operation algorithms, by revealing low-rank structures and leveraging weakly entangled, scale-separated MPS-based function representations. As a result, cross approximation algorithms offer significantly improved computational scaling and flexibility, making the (Q)TCI library suite \cite{Fernandez2024} highly attractive for a wide range of advanced applications. Among them the following are definitely worth mentioning (with their respective achievements): high-order real-time nonequilibrium Schwinger-Keldysh perturbation expansions \cite{Fernandez2022} (integral convergence improved from $1/\sqrt{N_{\text{func\_eval}}}$ to $1/N_{\text{func\_eval}}^2$), multi-dimensional function minimization and quantized reinforcement learning \cite{Sozykin2022} (outperfoming standard gradient-free methods in number of function evaluations and execution time), computation of Brilloin zone integrals for topological invariants evaluation \cite{Ritter2024} (exponential to polynomial-order scaling of integration costs with respect to the simulation parameters), compact tensorization of atomic orbitals bases with high accuracy \cite{Jolly2024} (error on g.s. energy of \ce{H2} improved by 85\% w.r.t. double zeta calculation), (speed-up of) multi-assets Fourier transform-based European option pricing \cite{Sakurai2025}.

More generally the novel approach of Ref. \cite{Fernandez2022, Fernandez2024} to integrate \textit{quantics} tensor rebasing with \textit{tensor cross approximation} procedures -- i.e. Quantics Tensor Cross Interpolation (QTCI) -- opened up the possibility to features like \textit{super-high resolution} and \textit{sign problem-free integration} for multi-dimensional functions, while maintaining computational costs bounded to a polynomial increase. Although the traditional ``TT-toolset'' already allowed for a similar scaling in resources with truncation-based procedures to reduce tensor's rank, (Q)TCI progressively reveals the rank structure of the input tensor by adaptively increasing the number of tensor evaluations (more on this in \prettyref{chap:QTCI}) without any loss of information caused by truncation. For this reason QTCI can be viewed as an \emph{active–learning} algorithm
\cite{Settles2012}:  it probes the input tensor only at those configurations that most efficiently expose its low–rank structure, thereby minimising the number of function evaluations needed to reach a prescribed accuracy.

The class of multivariate continuous functions that admit a low–rank tensor representation within a tolerance~$\varepsilon$—the so-called \emph{$\varepsilon$-factorisable} functions \cite{Fernandez2022}—appears to be
very large, although a rigorous characterisation is still lacking.  In practice this broadness translates into a wide domain of applicability for QTCI. 

Despite its versatility, the original TCI routine can suffer from \emph{ergodicity} problems: it may stall on very sparse tensors, tensors with
discrete symmetries, or tensorised functions featuring sharp local peaks. Global pivoting and related heuristics \cite{Fernandez2024} mitigate these issues but do not always succeed, particularly when the tensor originates from \textit{ab initio} calculations for which no \emph{a priori} information is available. More in general, since all TCI algorithms involve sampling, none of them is fully immune against missing some features of the tensor of interest.

To address these issues we propose in this work a \emph{divide-et-impera} extension of TCI. Inspired from similar distributed TT frameworks \cite{Hiroshi2023,Ehrlacher2021}, the algorithm adaptively partitions the target tensor into smaller subtensors (“patches”), TCI compressing each with a fixed bond-dimension cap, and thus concentrates resources on the most challenging regions of configuration space while keeping the overall memory footprint under control. 

The same patch philosophy also alleviates other tensor-network bottlenecks. In particular, we demonstrate how a similar domain decomposition limits intermediate bond growth in MPO–MPO contractions, substantially reducing their computational cost.

\note{Cite website and implementation}{A very well structured implementation of the main TCI routine can be found in the $\julia$ library \texttt{TensorCrossInterpolation.jl} \cite{TensorCrossInterpolation.jl}.  for the rest of the work. Other smart functionalities including the quantics tensor dicretizetion + TCI are also provided by analogous libraries as documented in the \textit{tensor4all.org} website \cite{tensor4all.org}. This implementation will be used as base for construction of our newly implemented patched TCI routine.}

The manuscript structured as follows: Chapter~\ref{chap:QTCI} reviews the standard (Q)TCI algorithm, its canonical implementation, and representative applications. Chapter~\ref{chap:patching} introduces the patched(divide-et-impera) variant—\emph{patched (Q)TCI}—and analyses its theoretical cost and limitations. Chapter~\ref{chap:MPOcontr} extends the patch strategy to MPO–MPO contractions, presenting both greedy and adaptive distributed schemes with cost estimates. Chapter~\ref{chap:results} presents numerical benchmarks and showcases the new algorithms on selected problems centred on the single-impurity Anderson model.


% \note{Tensor Networks}{\textit{MPS or TT toolset} for reducing the rank of tensor object}

% \note{Introduction}{Sign problem-free integration, \textit{superhigh-resolution}, exponential reductions in computational costs (\textit{curse of dimensionality}) $\to$ low-rank = calculations in polynomial time with no truncation (common approach), uncover hidden structures}\\

% \noindent\note{Other methods}{High dimensional integrals and exponential complexity of QMB systems, Quantum Montecarlo shortcomings: sign problem and slow convergence ($1 / \sqrt{N}$), different from TN for wave function variational ansatz (DMRG)}\vspace{\baselineskip}\\



% For next chapter:

% \note{Tensor Cross Interpolation}{Main paper: \cite{Fernandez2024}, Main TCI ref: \cite{Oseledets2010}, function learning, \textit{maxvol} principle \cite{Dolgov2020}, exact if $\rank{TT} = \chi$ like MCI (see Naive Approach \cite{Fernandez2022}), \textit{rank-revealing} (only slow convergence for high rank); separation of pivot exploration and tensor update could be beneficial (see Monte Carlo \textit{space configuration enlargment}); SVD stability issues compared to prrLU}\\

% \noindent\note{Nice sentences}{ Tensor network methods offer a new appraoch to high-dimensional integration.TCI has a very peculiar position among other tensor network algorithms: it provides
% an automatic way to map a very large variety of physics and applied mathematics problems onto the MPS toolbox. }\\

% \noindent\note{General references}{\cite{Oseledets2010, Oseledets2011, Savostyanov2014, Savostyanov2011, Dolgov2020}}\\
% \note{Open questions}{Generality of $\epsilon$-factorizability property}

% \noindent \note{Open questions QTCI 
% }{When does a multivariate function admit low-rank representation? }

% \note{Patching}{Main reference \cite{Hiroshi2023}; \textit{reset mode} of TCI eliminates bad pivots that occur when the algorithm first explores regions where $\mathcal{T}$ is small and only later configurations that are larger $\to$ this could be sufficient but not enough hence patching; patched TCI benefints from \textit{global pivot updates} coming from previous TCI decompositions; patching helps solving ergodicity problems; }\\

% \noindent \note{General references }{\cite{Murray2024} (first mention of the patching+QTT scheme)}\\ 

% \noindent \note{Plot ideas}{Plot the ratio of sampled points (function calls) vs total points of the calculation. }\\ 

% \noindent \note{Nice sentences}{ Nevertheless,
% since all TCI algorithms involve sampling, none of them is fully immune against missing some features of the tensor of interest, as already discussed above.}







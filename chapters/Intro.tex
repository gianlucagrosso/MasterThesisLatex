\chapter{Introduction}
\label{chap:intro}
``Dispelling'' the \textit{curse of dimensionality} that ``hexes'' numerics' computations, has been a challenge of prime interest in different fields of science for many years. 
Tensor network techniques -- and in particular \textit{matrix product states} (MPSs) methodologies -- are a well-established \cite{Fannes1992,White1992,Schollwock2011,Vidal2003,VerstraeteCirac2004,vonDelftTNNotes,tensornetwork.org} and standard approach widely employed by the quantum many-body (QMB) physics community to mitigate such exponential increase of computational resources. Numerical simulations of high-dimensional QMB wavefunctions represent, in fact, an exceptionally challenging computational task if not properly addressed.

Meanwhile mathematicians have dedicated, as well, significant effort to target the same problem -- arising from multidimensional tensor approximation of continuous functions and numerical linear algebra computations \cite{Oseledets2009Intro, Kolda2009} -- ultimately converging towards a similar approach: \textit{tensor trains} (TTs)\footnote{From now on \textit{tensor train} and \textit{matrix product state} will be used interchangeably.} \cite{Oseledets2011}. 

The rising interest in tensor-network (TN) methods -- particularly those based on MPSs — has therefore led to the emergence of a standard ``MPS toolbox'' \cite{ttpylib, ITensors.jl, QSpace} and a well-defined catalogue of canonical TN applications \cite{Verstraete2008} (variational ansatz of QMB wavefunctions -- or DMRG -- among the most popular). For a long time, however, techniques for function approximation and manipulation have been laying beyond the traditional TN portfolio.


Classical numerical techniques for representing and manipulating functions--whether for integration, convolution, differentiation, or related tasks--have advanced considerably over the years \cite{Isaacson1994}, yet they remain hindered by significant constraints. Grid‐based or naive SVD–style tensor decompositions, for instance, confront the \emph{curse of dimensionality}: the memory and CPU time required to store and update a high‐resolution representation of a $\mN$-dimensional function grow exponentially with 
$\mN$. Stochastic approaches such as Monte Carlo and Quantum Monte Carlo fare no better in the high‐dimensional regime; their error decreases only algebraically with sample size and they are plagued by additional obstacles—most notably the “sign problem’’—that can render simulations impractical for large, strongly correlated systems \cite{Loh1990}. Together, these limitations leave many modern, high-dimensional applications beyond the reach of standard numerical methods\cite{Fernandez2022}.

The widespread and increasing interest in TN and MPSs methodologies among different fields has therefore facilitated the development of a pivotal\footnote{A rather playful wording choice given the context.} extension to the ``MPS toolbox'' in order to integrate the missing function representation capabilities: the Tensor Cross Interpolation (TCI) algorithm.

TCI attempts to address many of the above-mentioned limitations of standard function approximation and operation algorithms, by revealing low-rank structures and leveraging weakly entangled, scale-separated MPS-based function representations. As a result, cross approximation algorithms offer significantly improved computational scaling and flexibility, making the (Q)TCI library suite \cite{Fernandez2024} highly attractive for a wide range of advanced applications. Among them the following are definitely worth mentioning (with their respective achievements): high-order real-time nonequilibrium Schwinger-Keldysh perturbation expansions \cite{Fernandez2022} (integral convergence improved from $1/\sqrt{N_{\text{func\_eval}}}$ to $1/N_{\text{func\_eval}}^2$), multi-dimensional function minimization and quantized reinforcement learning \cite{Sozykin2022} (outperfoming standard gradient-free methods in number of function evaluations and execution time), computation of Brilloin zone integrals for topological invariants evaluation \cite{Ritter2024} (exponential to polynomial-order scaling of integration costs with respect to the simulation parameters), compact tensorization of atomic orbitals bases with high accuracy \cite{Jolly2024} (error on g.s. energy of \ce{H2} improved by 85\% w.r.t. double zeta calculation), (speed-up of) multi-assets Fourier transform-based European option pricing \cite{Sakurai2025}.

More generally the novel approach of Ref. \cite{Fernandez2022, Fernandez2024} to integrate \textit{quantics} tensor rebasing with \textit{tensor cross approximation} procedures -- i.e. Quantics Tensor Cross Interpolation (QTCI) -- opened up the possibility to features like \textit{super-high resolution} and \textit{sign problem-free integration} for multi-dimensional functions, while maintaining computational costs bounded to a polynomial increase. Although the traditional ``TT-toolset'' already allowed for a similar scaling in resources with truncation-based procedures to reduce tensor's rank, (Q)TCI progressively reveals the rank structure of the input tensor by adaptively increasing the number of tensor evaluations (more on this in \prettyref{chap:QTCI}) without any loss of information caused by truncation. For this reason \todo{Machine learning part}

However... \todo{ Here you talk about TCI limitations and patching}


\todo{Here you talk about  content of each chapter }

\note{Tensor Networks}{\textit{MPS or TT toolset} for reducing the rank of tensor object}

\note{Introduction}{Sign problem-free integration, \textit{superhigh-resolution}, exponential reductions in computational costs (\textit{curse of dimensionality}) $\to$ low-rank = calculations in polynomial time with no truncation (common approach), uncover hidden structures}\\

\noindent\note{Other methods}{High dimensional integrals and exponential complexity of QMB systems, Quantum Montecarlo shortcomings: sign problem and slow convergence ($1 / \sqrt{N}$), different from TN for wave function variational ansatz (DMRG)}\vspace{\baselineskip}\\



For next chapter:

\note{Tensor Cross Interpolation}{Main paper: \cite{Fernandez2024}, Main TCI ref: \cite{Oseledets2010}, function learning, \textit{maxvol} principle \cite{Dolgov2020}, exact if $\rank{TT} = \chi$ like MCI (see Naive Approach \cite{Fernandez2022}), \textit{rank-revealing} (only slow convergence for high rank); separation of pivot exploration and tensor update could be beneficial (see Monte Carlo \textit{space configuration enlargment}); SVD stability issues compared to prrLU}\\

\noindent\note{Nice sentences}{ Tensor network methods offer a new appraoch to high-dimensional integration.TCI has a very peculiar position among other tensor network algorithms: it provides
an automatic way to map a very large variety of physics and applied mathematics problems onto the MPS toolbox. }\\

\noindent\note{General references}{\cite{Oseledets2010, Oseledets2011, Savostyanov2014, Savostyanov2011, Dolgov2020}}\\
\note{Open questions}{Generality of $\epsilon$-factorizability property}

\noindent \note{Open questions QTCI 
}{When does a multivariate function admit low-rank representation? }

\note{Patching}{Main reference \cite{Hiroshi2023}; \textit{reset mode} of TCI eliminates bad pivots that occur when the algorithm first explores regions where $\mathcal{T}$ is small and only later configurations that are larger $\to$ this could be sufficient but not enough hence patching; patched TCI benefints from \textit{global pivot updates} coming from previous TCI decompositions; patching helps solving ergodicity problems; }\\

\noindent \note{General references }{\cite{Murray2024} (first mention of the patching+QTT scheme)}\\ 

\noindent \note{Plot ideas}{Plot the ratio of sampled points (function calls) vs total points of the calculation. }\\ 

\noindent \note{Nice sentences}{ Nevertheless,
since all TCI algorithms involve sampling, none of them is fully immune against missing some features of the tensor of interest, as already discussed above.}


\textsc{Small caps text}
{\HUGE{Huge text}}


Some different type of text:\hfill

Mixing \textbf{different series, \textsf{families}} and
\textsl{\texttt{shapes,}} \textsc{especially in one sentence,}
is usually \emph{highly inadvisable!}\textit{ it = emph}

Some modification 

\MakeTextUppercase{⟨text⟩}


% \begin{figure}
% 	\centering
% 	\includegraphics[width=.5\textwidth]{example-image-duck}
% 	\caption{A cool duck}
% 	\label{fig:figure1}
% \end{figure}



% \label{sec:outline}

% A cool list:
% \begin{enumerate}
% 	\item item 1
% 	\item item 2
% 	\item item 3
% \end{enumerate}

% \rem{Something}
%  \todo{I have to fix this}
%  \note{Note }{just a note}

% \subsubsection{Some section}
% Some text
% \begin{definition}[Some law]
%     Some details
% \end{definition}
% Thank you for the space
% \change{\note{Gianluca}{I have to do this}}

% \begin{mydef}[Title]
%     Some text
% \end{mydef}
    

% $\Tr (\Gamma)$







